% !TeX encoding = UTF-8
% !TeX spellcheck = de_DE

\documentclass[biblatex]{lni}

%% Schöne Tabellen mittels \toprule, \midrule, \bottomrule
\usepackage{booktabs}

%% Zu Demonstrationszwecken
\usepackage[]{blindtext}
\usepackage{fancyhdr}
\usepackage{acronym}
\usepackage{authblk}

\begin{document}

  \begin{center}
    \LARGE Reflektion
  \end{center}

  \section{Einleitungssatz}

  \section{Einleitung}

  \section{Hauptteil}

  \\ \textbf{Was habe ich an der Hochschule gelernt, was ich im Praxismodul anwenden konnte?} \\
  Das Modul \textit{Web-Applikationen} gab mir das generelle Verständnis für die Fallstudie.
  \textit{Programmiermethoden und Werkzeuge} gab mir Grundlagen für die Implementierung.

  \\ \textbf{Welche Themen wären an der Hochschule sinnvoll, die mir bei der Umsetzung des Praxismoduls gefehlt haben?} \\
  Der Umgang mit Dokumentationen und die Evaluierung von Technologien anhand von realen Projektanforderungen.

  \\ \textbf{Was habe ich im Praxismodul gelernt?} \\
  Die Evaluierung verschiedener Technologien anhand konkreter Kriterien und ein klarer,
  skalierbarer Projektaufbau sind entscheidend für den Projekterfolg und sparen Zeit und Komplikationen.

  \\ Folgend immer auf einer Skala von 1 (zufrieden) bis 5(unzufrieden) bewertet.
  \\ \textbf{Arbeitseinsatz im Praxismodul: 2}\\
  Ich habe mich besonders bei der Test-Implementierung bemüht,
  jedoch hätte die Gesamtvorbereitung besser sein können,
  insbesondere im Verständnis der Nutzwertanalyse zur Bewertung der Anforderungen.

  \\ \textbf{Betreuung durch die Hochschule: 3}\\
  Die Betreuung war gut, trotz gelegentlicher Terminfindungsschwierigkeiten.
  Ich erhielt stets hilfreiche Rückmeldungen.

  \\ \textbf{Betreuung durch das Unternehmen: 2}\\
  Die Betreuung im Unternehmen war minimal,
  aber der Zugriff auf alle notwendigen Ressourcen war von Anfang an gegeben und positiv.

  \section{Schluss}

  \section{Schlusssatz}

\end{document}
